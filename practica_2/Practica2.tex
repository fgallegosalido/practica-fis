\input{preambuloSimple.tex}


%----------------------------------------------------------------------------------------
%	TÍTULO Y DATOS DEL ALUMNO
%----------------------------------------------------------------------------------------

\title{	
\normalfont \normalsize 
\textsc{\textbf{Fundamentos de Ingeniería del Software (2016-2017)} \\ Grado en Ingeniería Informática \\ Universidad de Granada} \\ [25pt] % Your university, school and/or department name(s)
\horrule{2pt} \\[0.4cm] % Thin top horizontal rule
\huge Práctica 2 \\ % The assignment title
\horrule{2pt} \\[0.5cm] % Thick bottom horizontal rule
}

\author{Miguel Ángel Torres López \and Francisco José Ruiz Jiménez \and Francisco Gallego Salido \and Francisco Lopez Rodríguez} % Nombre y apellidos


\date{\normalsize\today} % Incluye la fecha actual

%----------------------------------------------------------------------------------------
% DOCUMENTO
%----------------------------------------------------------------------------------------

\begin{document}

\maketitle % Muestra el Título

\newpage %inserta un salto de página

\tableofcontents % para generar el índice de contenidos

\newpage

\listoffigures

\newpage

%----------------------------------------------------------------------------------------
%	Cuestión 1
%----------------------------------------------------------------------------------------

\section{Objetivos}

Esta sección tiene como objetivo presentar y definir los casos de uso necesarios para comprender mejor el sistema planteado en la práctica 1. Al glosario de términos, 
la descripción de actores y la especificación de requisitos ahora añadimos los casos de uso y la relación de estos con los actores.

\vspace{8mm}

\section{Jerarquía de casos de uso}
\subsection{Gestión de Usuarios}
\textbf{Descripción:}\\
Escenarios asociados con acciones sobre los socios.\\
\textbf{Casos de uso:}\\
	\begin{itemize}
		\item Registro de usuario.
		\item Baja de usuario.
		\item Modificar datos de usuario.
		\item Consultar perfil de usuario.
		\item Consultar reservas actuales.
		\item Consultar historial de reservas.
		\item Añadir viviendas o vehículos al perfil.
		\item Gestionar viviendas y vehículos.
	\end{itemize}
\subsection{Gestión de Solicitudes y Reservas}
\textbf{Descripción:}\\
Escenarios asociados con la gestión de solicitudes y reservas.\\
\textbf{Casos de uso:}\\
	\begin{itemize}
		\item Solicitar/Reservar vehículo o vivenda.
		\item Cancelar solicitud.
		\item Gestionar críticas y puntuaciones de solicitantes.
		\item Adjuntar mensajes a una solicitud.
	\end{itemize}
\subsection{Gestión de Ofertas}
\textbf{Descripción:}\\
Escenarios asociados con la gestión de ofertas de usuarios.\\
\textbf{Casos de uso:}\\
	\begin{itemize}
		\item Búsqueda de ofertas.
		\item Gestión de ofertas realizadas.
		\item	Consultar disponibilidad de ofertas.
		\item Gestionar críticas y puntuaciones de solicitantes.
		\item Publicación de ofertas de vehículos y viviendas.
		\item Sesgar ofertas por preferencias.
		\item Cancelar ofertas antes de la fecha de uso.											
	\end{itemize}

\section{Diagrama de paquetes}
\begin{figure}[H] 
\centering
\includegraphics[scale=0.7]{Diagrama_de_paquetes.png}  
\caption{Diagrama de paquetes} \label{fig:figura1}
\end{figure}

\section{Diagramas de casos de uso}
\subsection{Gestión de usuarios}
\subsection{Gestión de solicitudes y reservas}
\subsection{Gestión de ofertas}
\end{document}